\documentclass[10pt,letterpaper]{article}
\usepackage[latin1]{inputenc}
\usepackage{amsmath}
\usepackage{amsfonts}
\usepackage{amssymb}
\usepackage{graphicx}
\usepackage{hyperref}
\usepackage{listings}


\author{Andrew Rosen}
\title{Palindromes}
\date{}

\begin{document}

\maketitle

\begin{abstract}
	The purpose of this assignment is to familiarize you with stacks and queues.  The assignment is split into 2 tasks.  Extra Credit is available on the second page.
\end{abstract}



\section{Stacks and Queues}
The first part of the assignment is to take the Node, Stack, and Queue files and implement them (IE fill in the methods).  
Do not extend your class.
References are to be singly linked.  

\section{Palindrome Detection}
Part two requires you to create another class to implement a Palindrome detector that uses a Stack and a Queue.  
It must fulfill the following requirements:


\begin{enumerate}
	\item The program must loop to take input from the user.  The input is a single word.  If the word is quit, go to step 4.  Otherwise we look to see if the input is an Palindrome (lowercase and uppercase are considered the same letter)
	\item It must use both the Stack and the Queue you just wrote to detect Palindromes.  To do this, push each letter into the stack and enqueue/offer each letter to the queue. 
	\item  Output whether or not the inputed string is an Palindrome.
	\item Once the user types quit, print out all the detected Palindromes in reverse order, then terminate.  Use the proper data structure.
	\item You cannot use arrays.
\end{enumerate}


\newpage
\section{Extra Credit: 10 points}
Rather than detecting if a word is a palindrome, instead tell me if an entire String is a palindrome.  An entire String is a palindrome iff the string reads the same backwards to forwards, ignoring capitalization and punctuation. For example, if you fufill the extra credit, the following line should be detected as a Palindrome.

\begin{verbatim}
"Go hang a salami, I'm a lasagna hog"
\end{verbatim}
	
\end{document}
