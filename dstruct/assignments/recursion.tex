\documentclass[10pt,letterpaper]{article}
\usepackage[latin1]{inputenc}
\usepackage{amsmath}
\usepackage{amsfonts}
\usepackage{amssymb}
\usepackage{graphicx}
\usepackage{hyperref}
\usepackage{listings}
\usepackage{xskak}

\usepackage{DejaVuSansMono}

\usepackage{chessboard}

\lstset{language=Java, basicstyle= \ttfamily\small, showstringspaces=false} 


\author{Andrew Rosen}
\title{Recursion Puzzles}
\date{}

\begin{document}

\maketitle

\begin{abstract}
	In this lab you will implement recursive solutions to classic CS questions.
	One will be a chess problem, and the other is solving Sudoku.
	Each part is worth 75 points
%	Choose and complete two of the three following problems.
\end{abstract}


\section{Backtracking with Recursion  - Featuring Chess}
\textbf{Choose and complete one of the two following chess problems}.  These problems can be solved using the backtracking algorithm shown below.

\begin{lstlisting}
boolean solve(board,  pos){

	if( pos is such that there is nothing left to solve){
		return true;
	}
	
	for each possible choice {
		if(valid(choice)){
			mark board at pos with choice;
			if(solve(board, pos + 1) == true){
				return true;
			}
			unmark board at pos if needed, as choice was invalid
		}
	}
	clear any choices entered at pos on board;
	
	return false; // backtrack
}


\end{lstlisting}

\newpage
\subsection{The Eight Queens Problem}
Write a recursive method which solves the eight queens problem.  You must find a state where you can place eight queens on a chessboard such that no queen can capture another queen.  Queens can move and capture pieces in the same row, column, or any diagonal.


You may use an $ 8 \times 8$ \texttt{int[][]} array to represent your chess board.
An example solution is below.

\begin{center}
	
\chessboard[setwhite={Qa1,Qb7,Qc4,Qd6,Qe8,Qf2,Qg5,Qh3},showmover=false]
\end{center}

Hint: only one queen can be placed in each column. This means we can simplify the problem a bit by asking ``for this current column, which row can I put a queen on?''

\newpage 
\subsection{Knight's Tour (More Challenging)}
Write a program to solve the Knight's Tour.  
In the Knight's Tour, we place a Knight on the chess board and move him until he visits each square of the chess board exactly once.\footnote{Please do not sack Constantinople on your way to the answer.}
A knight moves two squares horizontally or vertically and then one square in the axis it did not move it, creating a sort of ``L'' shaped (see below).
A square counts as visited once the knight lands in it.



\begin{center}
	\chessboard[pgfstyle=knightmove,
	markmoves={c3-e2, c3-e4,c3-d5,c3-b5,c3-a4,c3-a2,c3-b1,c3-d1},
	showmover=false,
	setwhite={Nc3}
	]
	
	
\end{center}

You may start your knight anywhere you like.
Your output should be either the chess board, but with each square marked by a number to designate the order in which the square was visited, or by listing the moves the knight makes.
If you can figure out a better way to represent your answer, we are open to that too.





\newpage

\section{Sudoku}
Write a program that can solve a Sudoku puzzle.
You can put the puzzle you want to solve into your source code as a 2D integer array.
See the lecture videos on the homework for more details.

You can use the same base recursive-backtracking algorithm that was used in the 8 queens problem.


\subsection{Extra Credit: 10 points}
\href{https://projecteuler.net/problem=96}{Solve Project Euler Problem 96.  This entire line is a link to it.}
\end{document}
